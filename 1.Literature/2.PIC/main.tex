\documentclass[a4paper]{article}

% #############################################################################
% Preamble for Thesis-MSc in English or Portuguese
% Required Packages and commands
% --> Please Choose the MAIN LANGUAGE for the Thesis in package BABEL (below)
% !TEX root = ./main.tex
% #############################################################################
% Thesis-MSc
% Version 4, August 2022
% BY: Prof. Rui Santos Cruz, rui.s.cruz@tecnico.ulisboa.pt
% #############################################################################
%
% -----------------------------------------------------------------------------
% PACKAGES ucs, utf8x, babel, iflang:
% -----------------------------------------------------------------------------
% The 'ucs' package provides support for using UTF-8 in LaTeX documents. 
% However in most situations it is not required.
\usepackage{ucs}
% The 'utf8' package contains support for using UTF-8 as input encoding. 
\usepackage[utf8]{inputenc}
\usepackage[T1]{fontenc}
% The 'babel' package may correct some hyphenation issues of LaTeX. 
% Select your MAIN LANGUAGE for the Thesis with the 'main=' option.
\usepackage[main=english,portuguese]{babel}
% The 'iflang' package is used to help determine the language being used. 
\usepackage{iflang}


% -----------------------------------------------------------------------------
% PACKAGE scrbase:
% -----------------------------------------------------------------------------
% The 'scrbase' package is used to help redefining document structure.
\usepackage{scrbase}
% -----------------------------------------------------------------------------
% PACKAGE mathtools, amsmath, amsthm, amssymb, amsfonts, nicefrac:
% -----------------------------------------------------------------------------
% These packages are typically required. 
% Among many other things they add the possibility to put symbols in bold
% by using \boldsymbol (not \mathbf); defines additional fonts and symbols;
% adds the \eqref command for citing equations.
\usepackage{mathtools, amsmath, amsthm, amssymb, amsfonts}
\usepackage{nicefrac}
%
% -----------------------------------------------------------------------------
% PACKAGE tikz:
% -----------------------------------------------------------------------------
% Tikz  for creating graphics programmatically.
\usepackage{tikz}
\usetikzlibrary{shapes.geometric, arrows, positioning}
% -----------------------------------------------------------------------------
% PACKAGES array, booktabs, multirow, colortbl, spreadtab:
% -----------------------------------------------------------------------------
% These packages are most usefull for advanced tables. 
% 'multirow' allows to join rows throuhg the command \multirow which works
% similarly with the command \multicolumn.
% The 'colortbl' package allows to color the table (foreground and background)
% The package 'booktabs' provide some additional commands to enhance
% the quality of tables
% The 'longtable' package is only required when tables extend beyond the length
% of one page, which typically does not happen and should be avoided
\usepackage{array}
\usepackage{booktabs}
\usepackage{multirow}
\usepackage{colortbl}
\usepackage{spreadtab}
\usepackage{longtable}
\usepackage{pdflscape}
\usepackage{float}
%
% -----------------------------------------------------------------------------
% PACKAGES graphicx, subfigure:
% -----------------------------------------------------------------------------
% The package 'graphicx' supports formats PNG and JPG.
% Package 'subfigure' allows to place figures within figures with own caption. 
% For each of the subfigures use the command \subfigure.
\usepackage{graphicx}
\usepackage[hang,small,bf,tight]{subfigure}
%
% -----------------------------------------------------------------------------
% PACKAGE caption:
% -----------------------------------------------------------------------------
% The 'caption' package offers customization of captions in floating 
% environments such figure and table
% \usepackage[hang,small,bf]{caption}
\usepackage[format=hang,labelfont=bf,font=small]{caption} 
% the following customization adds vertical space between caption and the table
\captionsetup[table]{skip=10pt}
%
% -----------------------------------------------------------------------------
% PACKAGE algorithmic, algorithm, algorithm2e:
% -----------------------------------------------------------------------------
% These packages are required if you need to describe an algorithm.
% The preference is for using 'algorithm2e'
%\usepackage{algorithmic}
%\usepackage[chapter]{algorithm}
\usepackage[ruled,vlined,algochapter,norelsize,\languagename]{algorithm2e}
%
% -----------------------------------------------------------------------------
% PACKAGE listings
% -----------------------------------------------------------------------------
% These packages are required if you need to list code snippets.
\usepackage{listings}
% Nicely syntax highlighted m-code in LaTeX documents with stylefile mcode.sty
% http://www.mathworks.com/matlabcentral/fileexchange/8015-m-code-latex-package
\usepackage[numbered]{mcode}
%
% -----------------------------------------------------------------------------
% Re-define listings captions and titles based on language.
\newcaptionname{portuguese}{\lstlistingname}{Listagem} % Listings CAPTIONS
\newcaptionname{portuguese}{\lstlistlistingname}{Listagens} % LIST of LISTINGS
%
% -----------------------------------------------------------------------------
% PACKAGE csquotes
% -----------------------------------------------------------------------------
% Quotation helper package
\usepackage{csquotes}
%
% -----------------------------------------------------------------------------
% PACKAGE todonotes
% -----------------------------------------------------------------------------
% Create TODO Notes in text
% The notes can be made invisible by just using the 'disable' option:
\usepackage[textwidth=2cm, textsize=small]{todonotes}
%\usepackage[textwidth=2cm, textsize=small, disable]{todonotes}
\setlength{\marginparwidth}{2cm}
%
% -----------------------------------------------------------------------------
% PACKAGE changes
% -----------------------------------------------------------------------------
% Track changes in document (changes in pdf preview).
%% Use "final" option to make all tracking markups invisible.
%\usepackage[authormarkup=superscript,authormarkuptext=id,markup=underlined,ulem={ULforem,normalbf},final]{changes}
\usepackage[authormarkup=superscript,authormarkuptext=id,markup=underlined,ulem={ULforem,normalbf}]{changes}
% commands:
% \added[id=xx]{text}
% \deleted[id=xx]{text}
% \replaced[id=xx]{deleted text}{added text}
% -----------------------------------------------------------------------------
% PACKAGES xcolor, color
% -----------------------------------------------------------------------------
% These packages are required for list code snippets.
\usepackage{xcolor}
\usepackage{color}
% The following special color definitions are used in the IST Thesis
\definecolor{forestgreen}{RGB}{34,139,34}
\definecolor{orangered}{RGB}{239,134,64}
\definecolor{lightred}{rgb}{1,0.4,0.5}
\definecolor{orange}{rgb}{1,0.45,0.13}	
\definecolor{darkblue}{rgb}{0.0,0.0,0.6}
\definecolor{lightblue}{rgb}{0.1,0.57,0.7}
\definecolor{gray}{rgb}{0.4,0.4,0.4}
\definecolor{lightgray}{rgb}{0.95, 0.95, 0.95}
\definecolor{darkgray}{rgb}{0.4, 0.4, 0.4}
\definecolor{editorGray}{rgb}{0.95, 0.95, 0.95}
\definecolor{editorOcher}{rgb}{1, 0.5, 0} % #FF7F00 -> rgb(239, 169, 0)
\definecolor{chaptergrey}{rgb}{0.6,0.6,0.6}
\definecolor{editorGreen}{rgb}{0, 0.5, 0} % #007C00 -> rgb(0, 124, 0)
\definecolor{olive}{rgb}{0.17,0.59,0.20}
\definecolor{brown}{rgb}{0.69,0.31,0.31}
\definecolor{purple}{rgb}{0.38,0.18,0.81}
%
% -----------------------------------------------------------------------------
% PACKAGE setspace:
% ----------------------------------------------------------------------------
% Provides support for setting the spacing between lines in a document. 
% Package options include single spacing, one half spacing, and double spacing. 
% Alternatively the spacing can be changed as required with:
% \singlespacing, \onehalfspacing, and \doublespacing commands
\usepackage{setspace}
%
% -----------------------------------------------------------------------------
% PACKAGE paralist
% -----------------------------------------------------------------------------
% This package provides the 'inparaenum' environment for inline lists
\usepackage{paralist}
% usage:
% \begin{inparaenum}[(a)]
% \item bla
% \item bla, bla
% \end{inparaenum}
% -----------------------------------------------------------------------------
% PACKAGE cite:
% -----------------------------------------------------------------------------
% The 'cite' package will result in citation numbers being automatically
% sorted and properly "ranged". i.e.,
% [1], [2], [5]--[7], [9]
\usepackage{cite}
%
% -----------------------------------------------------------------------------
% PACKAGE acronym:
% -----------------------------------------------------------------------------
% The package 'acronym' garantees that all acronyms definitions are 
% given at the first usage. 
% IMPORTANT: do not use acronyms in titles/captions; otherwise the definition 
% will appear on the table of contents.
\usepackage[printonlyused]{acronym}
%
% -----------------------------------------------------------------------------
% PACKAGE hyperref
% -----------------------------------------------------------------------------
% Set links for references and citations in document
\usepackage{hyperref}
% pre-configuration of hyperref
\hypersetup{ colorlinks=true,
             citecolor=cyan,
             linkcolor=darkgray,
             urlcolor=teal,
             breaklinks=true,
             bookmarksnumbered=true,
             bookmarksopen=true,
}
%
% -----------------------------------------------------------------------------
% PACKAGE url:
% -----------------------------------------------------------------------------
% Provides better support for handling and breaking URLs.
\usepackage{url} 
%
% -----------------------------------------------------------------------------
% PACKAGE Cleveref:
% -----------------------------------------------------------------------------
% Clever Referencing of document parts
% use \Cref[], or \cref[] for referencing items (Figures, Tables, Algorythms, 
% Equations, Chapters, Sections, etc. No need to write the Name of the item. 
% Note: portuguese is supported through "brazilian" option
\usepackage[\IfLanguageName{english}{english}{brazilian}]{cleveref}
%
% -----------------------------------------------------------------------------
% PACKAGE enumitem:
% -----------------------------------------------------------------------------
%For enhanced enumeration of lists
%\usepackage{enumitem}
\usepackage[shortlabels]{enumitem}
\setlist[description]{leftmargin=\parindent,labelindent=\parindent,itemsep=1pt,parsep=0pt,topsep=0pt}
%
% -----------------------------------------------------------------------------
% PACKAGE glossaries-extra:
% -----------------------------------------------------------------------------
% Allows creating a list of terms with definitions for those terms
\usepackage[xindy,nopostdot]{glossaries}

% Special style to print trailing dots before locations list
\makeatletter
\newglossarystyle{mystyle}{%
  \setglossarystyle{altlist}%
  \renewenvironment{theglossary}{%
  \begin{description}[style=standard,labelindent=0pt,itemsep=5pt]%
  }%
  {\end{description}}
  \renewcommand*{\glossentry}[2]{%
    \item[\glsentryitem{##1}%
      \glstarget{##1}{\glossentryname{##1}}]%
      \mbox{}\par\nobreak\@afterheading
      \glossentrydesc{##1}\glspostdescription
      {\def\hfill{\hskip 25pt plus 3fill}\dotfill\mbox{ ##2}}%
  }%
  \renewcommand{\glsgroupskip}{}%
}
\makeatother
\setglossarystyle{mystyle}

\makeglossaries

\newglossaryentry{maths}
{
    name=mathematics,
    description={Mathematics is what mathematicians do}
}

\newglossaryentry{LaTeX}
{
    name=LaTeX,
    description={LaTeX It is a mark up language specially suited for scientific documents as it can correctly format documents with all the typographical rules}
}


\newglossaryentry{formula}
{
    name=formula,
    description={A mathematical expression}
}
% #############################################################################
% GLOBAL FORMATTING OF THE THESIS DOCUMENT before using FANCY stuff
% Load Titlepage definition
\usepackage{./cover-titlepage}

% Set paragraph counter to alphanumeric mode
% DO NOT CHANGE these lines, otherwise the document will not respect the Guide
\renewcommand{\theparagraph}{\Alph{paragraph}~--}
\hoffset 0in
\voffset 0in
\oddsidemargin 0 cm
\evensidemargin 0 cm
\marginparsep 0in
\topmargin -0.25cm
\textwidth 16 cm
\textheight 22.4 cm
\makeatletter
% package indentfirst says \let\@afterindentfalse\@afterindenttrue
% and we revert this modification, reinstating the original definitio
% of \@afterindentfalse
\def\@afterindentfalse{\let\if@afterindent\iffalse}
\makeatother
% -----------------------------------------------------------------------------
% PACKAGE fancyhdr:
% -----------------------------------------------------------------------------
% The fancyhdr macro package allows to customize page headers and footers.
\usepackage{fancyhdr}
\pagestyle{fancy}
\renewcommand{\chaptermark}[1]{\markboth{\thechapter.\ #1}{}}
\renewcommand{\sectionmark}[1]{\markright{\thesection\ #1}}
\fancyhf{}
%#########################################################
% Choose the positioning of the Page Number
% Only on the Center side [C]
\fancyfoot[C]{\bfseries\thepage}
% Only on the Right side [R]
%\fancyfoot[R]{\bfseries\thepage}
% In case of Double sided printing, numbers are printed on Left (even pages) and on Right (odd pages)
%\fancyfoot[LE,RO]{\bfseries\thepage}
%#########################################################
\renewcommand{\headrulewidth}{0.0pt}
\renewcommand{\footrulewidth}{0.0pt}
\addtolength{\headheight}{2pt} % make space for the rule

\fancypagestyle{plain}{%
   \renewcommand{\headrulewidth}{0pt} % and the line
   \renewcommand{\footrulewidth}{0pt}
}
\fancypagestyle{blank}{%
   \renewcommand{\headrulewidth}{0pt} % and the line
   \renewcommand{\footrulewidth}{0pt}
}
\fancypagestyle{abstract}{%
   \renewcommand{\headrulewidth}{0pt}
   \renewcommand{\footrulewidth}{0.0pt}
}
\fancypagestyle{document}{%
	\renewcommand{\headrulewidth}{0.5pt}
	\renewcommand{\footrulewidth}{0.5pt}
	\addtolength{\headheight}{2pt} % make space for the rule
}
\setcounter{secnumdepth} {5}
\setcounter{tocdepth} {5}
\renewcommand{\thesubsubsection}{\thesubsection.\Alph{subsubsection}}
\renewcommand{\subfigtopskip}{0.3 cm}
\renewcommand{\subfigbottomskip}{0.2 cm}
\renewcommand{\subfigcapskip}{0.3 cm}
\renewcommand{\subfigcapmargin}{0.2 cm}
%
% -----------------------------------------------------------------------------
% PACKAGE minitoc:
% -----------------------------------------------------------------------------
% Package 'minitoc' creates a mini-table of contents (a “minitoc”) at 
% the beginning of each chapter of a document.
% This packages are required for the \fancychapter configuration
\usepackage{minitoc}
\setcounter{minitocdepth}{1}
\setlength{\mtcindent}{24pt}
\renewcommand{\mtcfont}{\small\rm}
\renewcommand{\mtcSfont}{\small\bf}
\renewcommand*{\kernafterminitoc}{\kern0.\baselineskip\kern0.ex}
\mtcselectlanguage{\languagename} 
% Now prepare the MINITOC
\def\boxedverbatim{%
  \def\verbatim@processline{%
    {\setbox0=\hbox{\the\verbatim@line}%
    \hsize=\wd0 \the\verbatim@line\par}}%
  \@minipagetrue%%%DPC%%%
  \@tempswatrue%%%DPC%%%
  \setbox0=\vbox\bgroup\vspace*{0.2cm}\footnotesize\verbatim
}
\def\endboxedverbatim{%
  \endverbatim
  \unskip\setbox0=\lastbox %%%DPC%%%
  \hspace*{0.2cm}
  \vspace*{-0.2cm}
  \egroup
  \fbox{\box0}% <<<=== change here for centering,...
}
% Now prepare the CHAPTER Number
\newcommand*{\chapnumfont}{%
%   \usefont{T1}{\@defaultcnfont}{b}{n}\fontsize{100}{130}\selectfont%
  \usefont{T1}{pbk}{b}{n}
  \fontsize{130}{100}
  \selectfont
  \color{chaptergrey}
}
\makeatletter
\def\@makechapterhead#1{%
  \vspace*{50\p@}%
  {\parindent \z@ \raggedright \normalfont
    {\chapnumfont\ifnum \c@secnumdepth >\m@ne
%         \huge\bfseries \@chapapp\space \thechapter
        \raggedleft\bfseries \thechapter
        \par\nobreak
        \vskip 20\p@
    \fi}
    \interlinepenalty\@M
    {\raggedleft\Huge \bfseries #1\par\nobreak}
    \vskip 40\p@
  }}
\makeatother
% Now put it all together as a command \fancychapter
\newcommand{\fancychapter}[1]{\chapter{#1}\minitoc\bigskip}
%
% #############################################################################
% ADDITIONAL COMMANDS AND CONFIGURATIONS
% #############################################################################
% This commmand allows to place horizontal lines with a custom width... 
% replaces the standard hline command
\newcommand{\hlinew}[1]{%
  \noalign{\ifnum0=`}\fi\hrule \@height #1 \futurelet
   \reserved@a\@xhline}
%   
% -----------------------------------------------------------------------------
% This command defines some marks... USEFUL FOR TABLES.
\def\Mark#1{\raisebox{0pt}[0pt][0pt]{\textsuperscript{\footnotesize\ensuremath{\ifcase#1\or *\or \dagger\or \ddagger\or%
    \mathsection\or \mathparagraph\or \|\or **\or \dagger\dagger%
    \or \ddagger\ddagger \else\textsuperscript{\expandafter\romannumeral#1}\fi}}}}
%


% -----------------------------------------------------------------------------
% The following configurations are used for LISTINGS of certain languages
\lstdefinestyle{XML} {
	language=XML,
	extendedchars=true, 
	breaklines=true,
	breakatwhitespace=true,
	emph={},
	emphstyle=\color{red},
	basicstyle=\small,
	xleftmargin=17pt,
	columns=fullflexible,
	commentstyle=\color{gray}\upshape,
	morestring=[b][\color{brown}]",
	morecomment=[s]{<?}{?>},
	morecomment=[s][\color{forestgreen}]{<!--}{-->},
	keywordstyle=\color{orangered},
	stringstyle=\ttfamily\color{black},
	% stringstyle=\ttfamily\color{black}\normalfont,
	tagstyle=\color{blue},
	% tagstyle=\color{darkblue}\bf,
	morekeywords={asn,action,addrType,abilityNAT,audioSampleRate,audiChannels,,bandwidth,bitmapSize,bitRate,connection,codecs,concurrentLinks,dependency,duration,frameRate,from,height,ip,id,lang,mimeType,onlineTime,peerMode,port,priority,peerProtocol,property,release,to,tier,type,transactionID,url,uploadBWlevel,version,width},
	otherkeywords={attribute,xmlns,schemaLocation,PresentationType,availabilityStartTime,availabilityEndTime,minimumUpdatePeriod,minBufferTime,UpdateTime},
}
% ----------------------------------------------------------------------------
\lstdefinelanguage{Assembler}{
	morecomment=[l];,
	keywords={ADD,ADDC,SUB,SUBB,CMP,MUL,DIV,MOD,NEG,AND,OR,NOT,XOR,TEST,BIT,SET,EI,EI0,EI1,EI2,EI3,SETC,EDMA,CLR,DI,DI0,DI1,DI2,DI3,CLRC,SHR,SHL,SHRA,SHLA,ROR,ROL,RORC,ROLC,MOV,MOVB,MOVBS,MOVP,MOVL,MOVH,SWAP,PUSH,POP,JZ,JNZ,JN,JNN,JP,JNP,JC,JNC,JV,JNV,JEQ,JNE,JLT,JLE,JGT,JGE,JA,JAE,JB,JBE,JMP,CALL,CALLF,RET,RETF,SWE,RFE,NOP},
	morekeywords={EQU,TABLE,WORD,STRING,PLACE},
} 
% ----------------------------------------------------------------------------
\lstdefinestyle{coloredASM}{
	language=Assembler,
	extendedchars=false,
	breaklines=true,
	tabsize=2,
	numberstyle=\tiny,
	numbers=left,
	breakatwhitespace=true,
	emph={},
	emphstyle=\color{red},
	fontadjust=true,
	basicstyle=\small\ttfamily,
	% basicstyle=\footnotesize\ttfamily,
	columns=fixed,
	xleftmargin=17pt,
	framexleftmargin=17pt,
	framexrightmargin=5pt,
	framexbottommargin=4pt,
	commentstyle=\color{forestgreen}\upshape,
	morestring=[b][\color{brown}]",
	keywordstyle=\color{darkblue},
	stringstyle=\ttfamily\color{black},
	literate={á}{{\'a}}1 {ã}{{\~a}}1 {â}{{\^a}}1 {é}{{\'e}}1 {É}{{\'E}}1 {ê}{{\^e}}1 {õ}{{\~o}}1 {ó}{{\'o}}1 {í}{{\'i}}1 {ç}{{\c{c}}}1 {Ç}{{\c{C}}}1,
}    
% ----------------------------------------------------------------------------
\lstdefinelanguage{CSS}{
	sensitive=true,
	morecomment=[l]{//},
	morecomment=[s]{/*}{*/},
	morestring=[b]',
	morestring=[b]",
	alsoletter={:},
	alsodigit={-},
	keywords={color,background-image:,margin,padding,font,weight,display,position,top,left,right,bottom,list,style,border,size,white,space,min,width, transition:, transform:, transition-property, transition-duration, transition-timing-function}
}
% ----------------------------------------------------------------------------
% JavaScript
\lstdefinelanguage{JavaScript}{
	morecomment=[s]{/*}{*/},
	morecomment=[l]//,
	morestring=[b]",
	morestring=[b]',
	morekeywords={typeof, new, true, false, catch, function, return, null, catch, switch, var, if, in, while, do, else, case, break}
}
% ----------------------------------------------------------------------------
\lstdefinelanguage{HTML5}{
	language=html,
	sensitive=true,	
	alsoletter={<>=-},	
	morecomment=[s]{<!-}{-->},
	tag=[s],
	otherkeywords={
	% General
	>,
	% Standard tags
	<!DOCTYPE,
	</html, <html, <head, <title, </title, <style, </style, <link, </head, <meta, />,
	% body
	</body, <body,
	% Divs
	</div, <div, </div>, 
	% Paragraphs
	</p, <p, </p>,
	% scripts
	</script, <script,
	% More tags...
	<canvas, /canvas>, <svg, <rect, <animateTransform, </rect>, </svg>, <video, <source, <iframe, </iframe>, </video>, <image, </image>, <header, </header, <article, </article},
	ndkeywords={
	% General
	=,
	% HTML attributes
	charset=, src=, id=, width=, height=, style=, type=, rel=, href=,
	% SVG attributes
	fill=, attributeName=, begin=, dur=, from=, to=, poster=, controls=, x=, y=, repeatCount=, xlink:href=,
	% properties
	margin:, padding:, background-image:, border:, top:, left:, position:, width:, height:, margin-top:, margin-bottom:, font-size:, line-height:,
	% CSS3 properties
	transform:, -moz-transform:, -webkit-transform:,
	animation:, -webkit-animation:,
	transition:,  transition-duration:, transition-property:, transition-timing-function:,
	}
}
% ----------------------------------------------------------------------------
\lstdefinestyle{htmlcssjs} {%
	% General design
	backgroundcolor=\color{editorGray},
		fontadjust=true,
	basicstyle=\small\ttfamily,   
	frame=b,
	% line-numbers
	xleftmargin={0.75cm},
	numbers=left,
	stepnumber=1,
	firstnumber=1,
	numberfirstline=true,	
	% Code design
	identifierstyle=\color{black},
	keywordstyle=\color{blue}\bfseries,
	ndkeywordstyle=\color{editorGreen}\bfseries,
	stringstyle=\color{editorOcher}\ttfamily,
	commentstyle=\color{brown}\ttfamily,
	% Code
	language=HTML5,
	alsolanguage=JavaScript,
	alsodigit={.:;},	
	tabsize=2,
	showtabs=false,
	showspaces=false,
	showstringspaces=false,
	extendedchars=true,
	breaklines=true,
	% German umlauts
	literate=%
	{Ö}{{\"O}}1
	{Ä}{{\"A}}1
	{Ü}{{\"U}}1
	{ß}{{\ss}}1
	{ü}{{\"u}}1
	{ä}{{\"a}}1
	{ö}{{\"o}}1
}
% ----------------------------------------------------------------------------
\lstdefinestyle{py} {%
	language=python,
	literate=%
	*{0}{{{\color{lightred}0}}}1
	{1}{{{\color{lightred}1}}}1
	{2}{{{\color{lightred}2}}}1
	{3}{{{\color{lightred}3}}}1
	{4}{{{\color{lightred}4}}}1
	{5}{{{\color{lightred}5}}}1
	{6}{{{\color{lightred}6}}}1
	{7}{{{\color{lightred}7}}}1
	{8}{{{\color{lightred}8}}}1
	{9}{{{\color{lightred}9}}}1,
	basicstyle=\small\ttfamily,
	numbers=left,
	% numberstyle=\tiny,
	% stepnumber=2,
	numbersep=5pt,
	tabsize=4,
	extendedchars=true,
	breaklines=true,
	keywordstyle=\color{blue}\bfseries,
	frame=b,
	commentstyle=\color{brown}\itshape,
	stringstyle=\color{editorOcher}\ttfamily,
	showspaces=false,
	showtabs=false,
	xleftmargin=17pt,
	framexleftmargin=17pt,
	framexrightmargin=5pt,
	framexbottommargin=4pt,
	backgroundcolor=\color{lightgray},
	showstringspaces=false,
}

% -----------------------------------------------------------------------------
% The official name of the course/degree. Automatic for MEIC. It depends on language
%
\IfLanguageName{english}
		{\degree{Computer Science and Engineering}}
		{\degree{Engenharia Informática e de Computadores}}


\begin{document}

%%%%%%%%%%%%%%%%
% TITLE & NAME %
%%%%%%%%%%%%%%%%
\selectlanguage{english}
\title{Maximum Variance Unfolding on Disjoint Manifolds} % <-- ACTION: WRITE your project title here
\author{João André Roque Costa}
\istid{99088}
\email{joaoarcosta@tecnico.ulisboa.pt}
\advisor{Francisco Melo}
\coadvisor{João Tomás Brazão Caldeira} 

% To make the title
\maketitle
\thispagestyle{empty}

%%%%%%%%%%%%
% ABSTRACT %
%%%%%%%%%%%%
% To remove indentation before abstract
\setlength{\abstitleskip}{-\absparindent}
\begin{abstract}
    Dimensionality reduction is a field in constant evolution. However, even though there are methods that excel at computing linear relation over datasets, the operation of linearising data points that relate non-linearly between them is a task that was not perfected yet.
    %Considering different methods to different scenarios, there are cases where the present approaches perform poorly.
    %On this study, we are going to analyse whether if it is possible for the present non-linear methods to preform accurately on datasets with vast discrepancies in data point density. More precisely, if maximum variance unfolding can preform well over disjoint manifolds.
    In this study we assess whether a powerful non-linear dimensionality reduction method, maximum variance unfolding, can be extended to datasets that lie on disjoint manifolds.
\keywords{Dimensionality Reduction, Non-linearity, Embedded Space, Low Dimension, Eigenvalue Decomposition}
\clearpage
\end{abstract}

%%%%%%%%%%%%%%%%%%%%%
% TABLE OF CONTENTS %
%%%%%%%%%%%%%%%%%%%%%
\tableofcontents
\thispagestyle{empty}
\clearpage

\section{Introduction}
% Esta cena é fixe porque é bue util, olha estas utilidades todas, mas tem uns problemas, que são estes ___, este estudo procura expandir a usabilidade desta mega ferramenta


%% This thing is important (DR)
Dimensionality reduction is a vast research field whose premise is to reduce the dimensionality of a dataset in some way that approximates the number of variables used to its intrinsic dimensionality.

%% And these methods (PCA, MDS) don't work well, so there are these methods (___), but the (global ones) are not the greatest when we need local properties.
Acknowledging that there is the possibility for points to be related in non-linear ways, common approaches that deal with linear relations, like Principal Component Analysis (PCA) \cite{pca} and Multidimensional Scaling (MDS), do not perform well over them. To accurately perform non-linear dimensional reduction (NLDR) it is necessary to consider non-linear methods, which can be classified into neighbourhood graph (NG)-based and global (e.g., PCA, autoencoders \cite{autoencoder, autoencoder-book}) methods. The latter do not achieve competitive results when local properties are important to maintain, while methods that make use of NGs allow for control over geometric properties of the data when performing dimensionality reduction.

%% To fix that, Here are some general development about this that has already been made.


%% This is the main problem why it fails (Disjoint components).
%Moreover, the subfield of non-linear dimensional reduction (NLDR) is lacking adequate tools to perform successfully over many key scenarios, being the disconnection between clusters of points a common deficiency for many methods.
While NLDR methods based on neighbourhood graphs (NG-NLDR) are the most adequate for finding representations that preserve geometric properties of the data, they are not easily applicable in the common scenarios where data lie on multiple manifolds or are sampled sparsely from the underlying manifold. In these cases, since we usually connect each point to its $k$-nearest neighbours, the formed NG may consist of multiple connected components. For example, if we imagine two clusters of points that are very distant from each other, it might take a large value of $k$ for them to become connected. However, as we will see, large values of $k$ (i.e., considering large neighbourhoods around each point) reduces NG-NLDR methods' ability to focus on local geometric properties. On the other hand, disconnected NGs result in NG-NLDR procedures that are ill-defined or yield bad results.

%% This is how it fails:
%The methods that most suffer from disjoint datasets are graph based NLDR methods. These, by computing each point's reduced position by analysing each one's neighbourhood, struggle on locations where the dataset is sparse, and thus each point's neighbourhood is hard to analyse. 
%From a more general view, the most common cause is the separation between classes, where there is little data to connect them, thus creating a clear separation between different groups of points.

%% Other attempts at fixing the fail:
To respond to this, there is a present need for methods to be robust to datasets of varying density. In \cite{inspiration, general-landmark, modern} are discussed some implementations to fix this problem, from creating connections between disjoint components whenever they are not connected, or reducing a component's points into a single reference one, to then, after NLDR, rebuild the entire dataset from these reference points.


%% Our approach to fixing the fail

In this study we propose an extension to a valuable dimensionality reduction method, maximum variance unfolding (MVU), in order to address its inapplicability to the common setting of disconnected NGs. We reiterate that these emerge when the data lie on multiple manifolds or when certain regions of the manifold are more sparsely sampled, both of which occur naturally in real datasets.

%% This paper is about...:
First, we are describing the process of dimensionality reduction, presenting different methods dedicated to different purposes and also explaining their reasoning.

More specifically, how some linear methods, with some tweaking, can solve non-linear datasets; some non-linear methods can reach good results, but others that reach more precise solutions are limited in their usability on natural data. With the presence of class separation, many methods, mainly graph-based, can't perform well due to the sparsity over areas of lower point density.

After that, we analyse how there are some extensions to the methods that simplify the input data by a smaller dataset and still can perform as well. We also present some extensions that adapt to sparse areas, but on some related methods other them MVU.
\section{Literature Review}

    % colour code:
    % red: about to drop out, unless it is requested to stay / some notes
    % orange: newly inserted annotations/doubts
    % olive: reformulated information in the need for a last review to be added
    % brown: to review later/somethings I need to study more to explain


    % \paragraph{Curse of dimensionality} Curse of dimensionality is declared whenever a data set consists of too many variables comparatively to its number of records. This phenomenon represents a problem because for each variable inserted into a set, the number of records needed to represent the possible values taken by those variables grows exponentially.

    \subsection{Convex Optimisation}

    \paragraph{Convex Sets}
    Given a set $S\in R^D$, it is considered convex if and only if for any pair of points, all points in the straight line between them are contained in the set.
    This is a simplification of the following condition:
    \begin{equation}
        (1- \alpha)x + \alpha y \in S,
    \end{equation}
    for all $x$,$y \in S$ and $\alpha \in [0, 1]$.
    
    
    \paragraph{Convex Functions}
    By definition, a function $f$ is convex if for any pair of values, $x$ and $y$, the value of the function is never higher than the line segment between these two points. This can be written with the following mathematical expression:
    \begin{equation}
        f(\alpha x + (1-\alpha)y) \le f(\alpha x) + f((1-\alpha)y),
    \end{equation}
    for all $x$,$y \in  $ dom$(f)$ and $\alpha \in [0, 1]$.
    
    There are cases where a function, and thus an optimisation problem, might be convex but take no global minimum like the case of the exponential function. On the other hand, a constant function would take infinite global minima. 
    
    \paragraph{Optimisation Problem}
    An optimisation problem consists of finding the optimal values of a set of variables $x$, possibly within a subset $\Omega\in D$ of their domain (which is represented by constraints), such that some function of interest $f$ is either maximized or minimised, depending on the problem. When the purpose of the problem is to pursue a maximization of the optimised function, it shall be denominated utility function, while when it is meant to be minimised it is called cost function. We call "variables of interest" the variables to be changed in order to reach the optimal function value. My suggestion for presenting the general form of an optimisation problem is:
    \begin{align}
        \min_{x\in D} \quad & f(x)\\
        \textrm{s.t.} \quad 
            & x\in\Omega, \text{ where }\Omega\subseteq D.
    \end{align}


    Each optimisation problem can be classified considering the nature of the function to optimise and the existence of the constraints.
    If a problem has any kind of constraint, it is declared as a Constrained Problem. If either the objective function is not convex, or the constraints define a non-convex set, then the whole problem is denoted as non-convex. This is, for a problem to be convex, all the functions to optimise and the constraints have to be convex.
    
    According to the differences in optimisation problem and types of constraints, the optimisation problems can be classified as:
    \begin{itemize}
        \item \textbf{Linear Programming (LP)} problems take an affine function as an objective function and linear constraints. They are computationally the simplest to solve, usually solvable using the simplex algorithm or the interior point method.
        \item \textbf{Quadratic Programming (QP)} problems involve the optimisation of a quadratic function, while also being limited to linear constraints. Depending on the convexity of the problem, a simple gradient method may reach the global minimum if it is an unconstrained problem, while the Karush–Kuhn–Tucker conditions may be used to successfully solve constrained ones. On non-convex situations when either the objective function or the constraint's set are non-convex, a gradient method would need the help of a global search extension to then be able to find the global minima.
        
        \item \textbf{Second Order Cone Programming (SOCP)} problems consist of the optimisation problems constrained by constraints that form a second order cone, these are formulated as $\| \bm{A}x+b \|_2 \le c^\top x +d$.
        
        \item \textbf{Semi-definite Programming (SDP)} problems represent the optimisation problems that are constrained by a linear matrix inequality of the form $x_1\bm{F}_1 + \ldots + x_n\bm{F}_n + \bm{G} \preceq 0$.
        %TODO: linear matrix inequalities
        % https://web.stanford.edu/~boyd/cvxbook/bv_cvxbook.pdf#page=182.55
    \end{itemize}

    \subsection{Dimensionality Reduction}
    %TODO: rewrite
    Dimensionality reduction is the process of reconstructing a given dataset in a space described by fewer dimensions. This is done by finding a projection from the higher to lower dimensional space, approximating the dimensions of the dataset in analysis to its intrinsic dimensions, the theoretical number of variables necessary to fully represent the data in observation.
    
    \subsubsection{Linear Methods}
    
    \paragraph{Principal Component Analysis (PCA)}
        The most commonly used is Principal Component Analysis \cite{pca} which, by comparing the linear relation between each pair of variables, creates a new set of variables in a linear subspace that retains as much variance as possible. Its simplified procedures are:
        
        Let's consider a data set with $n$ records represented by $D$ variables ($\bm{X}\in\mathbb{R}^{n\times D}$), which we want to reduce to $d$ variables ($\bm{Y}\in\mathbb{R}^{n\times d}$), where $d << D$.
        
        For PCA to compare and find the best $d$ variables composed of the original ones, it relates them by computing a covariance or correlation matrix which, importantly, centres each variable in its calculations.
        \begin{align}
            \text{covariance}(\bm{X}_i, \bm{X}_j) = \frac{\sum_k (\bm{X}_{ik} - \overline{\bm{X}_i})(\bm{X}_{jk} - \overline{\bm{X}_j})}{n}, &&
            \text{correlation}(\bm{X}_i, \bm{X}_j) = \frac{\text{covariance}(\bm{X}_i, \bm{X}_j)}{\sigma_{\bm{X}_i}\sigma_{\bm{X}_j}}.
        \end{align}
        %TODO: usually using the covariance matrix, the covariance matrix is formed from each pair of points 
        
        Usually using the covariance, the similarity matrix $\bm{\Sigma}\in\mathbb{R^{D\times D}}$ is constructed iteratively relating each pair of variables, but it can also be calculated matricially following: \\
        \begin{align}
            \bm{X'}_k &= \bm{X}_k - \overline{\bm{X}_k} \qquad, \forall k \in \{0, 1, \ldots, D\} \\
            \underbrace{\bm{\Sigma}}_{D\times D} &= \frac{\overbrace{\bm{X'}^\top}^{D\times n}\overbrace{\bm{X'}}^{n\times D}}{\bm{n}},
            \label{eq: Sigma}
        \end{align}
        where $\bm{X'}_k\in\mathbb{R}^n$ represents the $\bm{X}_k$ matrix centred by components. \\
        
        Now that we have the \textit{similarity matrix} $\bm{\Sigma}$, we look for a new orthogonal basis that maximises the variance of the dataset over the minimum components possible. To do this, since we are considering a symmetric and positive semi-definite matrix, performing an eigenvalue decomposition over the matrix $\bm{\Sigma}$ returns a set of orthogonal eigenvectors, and their eigenvalues, representing the orthogonal basis that maximises the dataset's variance.
        
        These, when analysed individually, represent each of the new directions of the new eigenbasis and the scale of the variance of the dataset over that particular direction.

        Note that this approach can be used over any type of similarity matrix, for example, any symmetric distance matrix between points can also be used, for example, the Gram matrix $\bm{K} = \bm{XX}^\top$. However, they need to be double-centred following the transformation:
        \begin{equation}
            k_{ij} = - \frac{1}{2}\left(d_{ij}^2 -\frac{1}{n} \sum_l d_{il}^2 - \frac{1}{n} \sum_l d_{lj}^2 + \frac{1}{n^2} \sum_{lm} d_{lm}^2 \right),
        \end{equation}
        where $d_{ij}$ represent the euclidean distance between $i$ and $j$, for example. % Note that other methods also have formulations comparable to PCA with covariance which do not need double-centring, such as MDS, Isomap, LEE, LE, H-LLE, MVU.
        
        Since the purpose is to reduce the dimensionality of the data set, we do that by selecting the $d$ new variables, i.e. eigenvectors, that correspond to the biggest eigenvalues, ending up with the new variables that display the most variance. The value of $d$ is usually set as a way to retain a certain percentage of the variance in the original data, usually $95-99\%$ of the variance.\\

        The operation to perform the eigenvalue decomposition is:
        The conditions that need to stand when performing the eigenvalue decomposition are: 
        \begin{equation}
            \bm{\Sigma V} = \bm{V\lambda}
            \quad \iff \quad
            \bm{\Sigma} = \bm{V\lambda V}^\top
            \label{eigenvalue decomposition},
        \end{equation}
        where $\bm{\lambda}\in\mathbb{R}^{d\times d}$ is the diagonal matrix of the $d$ biggest eigenvalues of $\bm{\Sigma}$, and $\bm{V}\in\mathbb{R}^{D\times d}$ is the column matrix of eigenvectors corresponding to $\lambda$'s eigenvalues. Importantly for these calculations, these two conditions are equivalent solely because $\bm{\Sigma}$ is symmetric also due to that, the eigenvector matrix $\bm{V}$ represent an orthogonal basis.

        To project the data points from the higher dimensional space to the lower one perform:
        \begin{equation}
            \bm{Y} = \bm{XV},
            \label{reduction mapping}
        \end{equation}
        where $Y\in\mathbb{R}^{n\times d}$ represent the data points in the embedding space (i.e., the PCA projections) and $\bm{X}\in\mathbb{X}^{n\times D}$ is the source matrix.
        
        Finally, we can reconstruct an approximation of the original data using the \textit{projection matrix} $\bm{VV}^\top$:
        \begin{equation}
            \bm{\hat{X}} = \underbrace{\bm{XV}}_{\text{PCA projs.}}\bm{V}^\top.
        \end{equation}
        
        We call the \textit{embeddings}, the representation of the original data points in the lower dimensional space, but in this particular method, they are also called the PCA projections.
        % also dizer que somamos a média se os dados não estiverem centrados
        If the original data were not already centred, the final step is to sum the mean of each variable to each point, returning the reconstructed data to the same relative position as the original. Note that, here, we operate over the reconstructed points in the data space, in $D$ dimensions.
    
    \paragraph{Multidimensional Scaling (MDS)} % MDS como problema de optimização

        Like PCA, Multidimensional Scaling (MDS) \cite{mds, mds-nonmetric} is a linear method that finds the best orientations for new variables from an eigenvalue decomposition, but this time the matrix to operate on can consider any form of distance between the original points and after the reduction.
        
        It can also be represented as an optimisation problem, where the objective is to minimise the difference of the distances between the two spaces, i.e. for a given point on the original space, we want its distance to any other point to be equal to the \textit{euclidean} distance between these points in the space after applying dimensionality reduction.
        
        MDS is another linear method that given a desired number of variables, maintains the distance between each pair of points the best it can. Like PCA, it can be solved in closed form.
        
        Analytically, the process is to build a centred inner-product matrix $\bm{B}\in\mathbb{R}^{n\times n}$, commonly considered a \textit{similarity matrix} (can be viewed as how much two given points are related). It is built from the distance matrix $\bm{D}\in\mathbb{R}^{n\times n}$ representing the distance between any two given points. $\bm{B}$ is then calculated:
        \begin{align} % source: https://rich-d-wilkinson.github.io/MATH3030/6-1-classical-mds.html#eq:defB
            \bm{B} = -\frac{1}{2}\bm{HD} \times \bm{DH},
            \label{MDS similarity matrix}
        \end{align}
        where $\bm{H}$ represents the centring matrix $\bm{H} = \mathbb{I}_n - \frac{1}{n}\bm{1}_n\bm{1}_n^\top$, fundamentally, $\bm{1}_n\bm{1}_n^\top$ represent an $n\times n$ matrix of 1's. Note that $\bm{D} \times \bm{D}$ represent the pairwise product (Hadamard product).
        
        After that, an eigenvalue decomposition can be performed over the matrix $\bm{B}$ and we can select the eigenvectors corresponding to the $d$ biggest eigenvalues. The procedure is the same as in PCA (Eq. \ref{eigenvalue decomposition}).
        
        MDS can be written as an optimisation problem:

        \begin{equation}
            \min_{\bm{Y}} \quad \sum_{ij} \left( 
            d_{ij} - \| \bm{y}_i - \bm{y}_j \|_2
            \right)^2,
            \label{MDS formulation problem}
        \end{equation}
        where $\bm{Y}=\begin{bmatrix}\bm{y}_1 & \cdots & \bm{y}_n\end{bmatrix}\in\R^{n\times d}$, and $d_{ij}$ represents a distance function between $\bm{x}_i$ and $\bm{x}_j$ (i.e., points in the original space).

        In \textit{Classical} MDS, we attempt to preserve euclidean distances in the embedding space, i.e., we solve:
        
        \begin{equation}
            \min_{\bm{Y}} \quad \sum_{ij} \left( 
            \|\bm{x}_i - \bm{x}_j \|_2 - \| \bm{y}_i - \bm{y}_j \|_2
            \right)^2
            \label{classical MDS formulation problem}
        \end{equation}

        In the same circumstances, since we know that $\bm{B}$ will be symmetric and positive semidefinite, we can directly calculate it from the original dataset:
        \begin{equation}
            \bm{B} = \left( \bm{HX} \right) \left( \bm{HX} \right)^\top,
        \end{equation}
        to further proceed with the closed-form approach.

        Classical MDS gives the exact same solution as PCA. As we will see in the next section, we may choose other distance metrics to account for nonlinearities in the data manifold.
        
        \paragraph{Observations}
        Although linear methods can be very useful and accurate on many linear cases, because they assume linearity between variables, when put to use on data sampled from non-linear manifolds they can't capture this complex relation between variables, so their performance decreases in function of the curvature of the manifold.

    \subsubsection{Isomap} 
        The Isomap \cite{isomap} reduction method can be seen as an extension of MDS where the distance matrix to be used is not calculated by Euclidean distances but rather by geodesic distances.

        
        %By assuming local linearity, Isomap's optimisation function is not about the Euclidean distance between points as Classical MDS is, but the geodesic distance instead. Fundamentally, aside from the distance calculations, they are very alike. \\
        
        %Mathematically, the geodesic distance represents the shortest distance between two points along a manifold. Since the manifold is approximated through a neighbourhood graph, we approximate geodesic distances between points as shortest paths along the graph, usually computed using the Dijkstra or Floyd-Warshall algorithms.
        % TODO: merge
        %but importantly, the path can only transit between points, so it is calculated as the smallest sum of Euclidean distances of a path between two points; its true value is regularly approximated from the Dijkstra or Floyd-Warshall algorithms.\\
        
        \textit{Local linearity} is an important and common assumption made by many non-linear methods where the curvature of the manifold in the small region around a given point can be considered linear, thus ignoring the curvature and allowing for the true distance between points to be approximated by the Euclidean distance. This property is crucial for geodesic distance calculations.

        The geodesic distance is the theoretical distance between two points along a manifold. In practice, we assume local linearity. This assumption enables us to create neighbourhoods of data points around each one from the Euclidean distance between them, this is, by calculating the Euclidean distance from a point to all the others and selecting only the closest ones, we create a \textit{neighbourhood graph}.
        Connecting neighbours in the neighbourhood graph and finding the shortest path between two given points is a good approximation of the geodesic distance. A usual approach for calculating it is using the Dijkstra or Floyd-Warshall algorithms. \\
        % TODO: geodesic distance?-> neighbourhood graph -> local linearity -> Dijkstra

        In general, we cannot expect euclidean distances between points in the data space to be proportional to distances along the data manifold. This downside is even more noticeable on distant points. To better relate points along a manifold, we may use geodesic distances, since they basically calculate the same Euclidean distance if only the manifold was approximately laying flat, overcoming the non-linearity by traversing the manifold along neighbour points.
        
        Here, the method builds a distance matrix $\bm{D}\in\mathbb{R}^{n\times n}$ representing the geodesic distance from a given point to each of the others. Using geodesic distances is what makes this method non-linear. To calculate geodesic distances along a graph, we can use the Dijkstra or Floyd-Warshall algorithms. The optimisation problem is formulated as in Eq. \ref{MDS formulation problem}, where the distance between points $\bm{x}_i$ and $\bm{x}_j$ is given by the approximated geodesic distance $d_{ij}$.
        
        Like MDS, we can solve the problem in closed form by squaring the distance matrices and centring them, like a PCA approach that uses a distance matrix instead. The approach is to calculate the similarity matrix $\bm{B}$ (as in Eq. \ref{MDS similarity matrix}) using (approximated) geodesic distances $d_{ij}$ rather than euclidean distances in the data space.

        Note that this geodesic distance function can also be viewed as a kernel for other methods, i.e., a transformation function/matrix relating the data points.
    
    \subsubsection{Locally Linear Embedding (LLE)}
        To mitigate MDS and Isomap's main flaw of giving too much focus to the distance between very far apart points instead of the closest and most important ones to optimise, the LLE \cite{lle} bases itself on the relation of each point with its neighbours.

        Because its process of reducing the dimensionality is based on translation, rotation, and rescaling, by creating a matrix of influences $\bm{W}$ between neighbouring points, it is then able to use this relation matrix to embed the data. The problem is represented as:
        \begin{equation}
            \min_{\bm{Y}} \quad \sum_i {\| \bm{y}_i - \sum_j w_{ij} \bm{y}_{ij} \|}^2,
        \end{equation}
        where ${w_i}_j \in [0,1]$ represents how much a point $j$ influences $i$'s position. Note that we only calculate influences between points that are $k$-nearest neighbors. This is how the weight matrix is calculated:
        \begin{align}
            \min_{\bm{W}} \quad &
            \sum_i {\| \bm{x}_i - \sum_j w_{ij} \bm{x}_{ij} \|}^2 \\
            \textrm{s.t.} \quad 
                & \sum_{j\in\mathcal{N}_i} w_{ij} = 1,\quad\forall i=1,\dotsc,n,
            \label{eq:lle-weights}
        \end{align}
        where $\mathcal{N}_i$ is the set of neighbours of $i$. Consequently, the rest of the $w_{ij}$ for points $j$ that are not in the neighbourhood of $i$ are left as 0, making $\bm{W}$ a sparse matrix.
        
        With this said, the problem minimises the difference between the current position of a given point and the position that its neighbours would pull it to, originally. This creates a trivial solution where all points coincide at the origin so, to fix this, the constraint $ \| y_i^{(k)} \|^2 = 1$ for all $k$ dimensions is added, preventing points from ending up in the origin.
        
        The optimal solution to the optimisation problem can be found following:
        \begin{align}
            \sum_i {\| \bm{y}_i - \sum_j \bm{w}_{ij} \bm{y}_{ij} \|}^2
                &= (\bm{Y}- \bm{WY})^\top(\bm{Y}- \bm{WY}) \\
                &= \bm{Y}^\top (\bm{I} - \bm{W})^\top(\bm{I} - \bm{W}) \bm{Y} \\
                &= (\bm{I} - \bm{W})^\top(\bm{I} - \bm{W}),
        \end{align}
        where $\bm{I}\in\mathbb{R}^{n\times n}$ is the identity matrix and $\bm{W}\in\mathbb{R}^{n\times n}$ is the reconstruction matrix.
        
        The resulting embedding matrix $\bm{Y}\in\R^{n\times n}$ can then be used as in Eq. \ref{eigenvalue decomposition} as a similarity matrix, where the selection of the $d$ smallest eigenvectors ends up with the usual eigenbasis for the embedded space.

    \subsubsection{Laplacian Eigenmaps (LE)}
        Like LLE, Laplacian Eigenmaps (LE) \cite{le} create a weight matrix $\bm{W}$ which quantifies the influence of each neighbour of a given point in its coordinates. This, while assuming that the specific point and its neighbours are in the same plane, that is, local linearity.

        Instead of minimising the distance to the theoretical weighted position that each point's neighbours are pulling it to, LE minimises the weighted distance to each neighbour directly:
        \begin{equation}
            \min_{\bm{Y}} \quad \sum_{ij} {\| \bm{y}_i - \bm{y}_j \|}^2 w_{ij},
        \end{equation}
        where $\bm{W}$ is a sparse matrix, that for each pair of neighbour points $i$ and $j$, ${w_i}_j$ is computed using the Gaussian kernel function:
        \begin{equation}
            w_{ij} = e^{-\frac{{\| \bm{y}_i - \bm{y}_j \|}^2}{2\sigma^2}}.
        \end{equation}
        This means that neighbour points in the high-dimensional space are put as close together as possible in the embedding space.
        
        To further solve the optimisation problem, the following diagonal degree matrix $\bm{M}\in\mathbb{R}^{n\times n}$ is built based on $m_{ii}=\sum_j \bm{w}_{ij}$, consisting of the sum of the weights at each point, note that they do not sum up to $1$ as in Eq. \ref{eq:lle-weights}.
        
        The problem can then be decomposed into:
        \begin{equation}
            \sum_{ij} {\| \bm{y}_i - \bm{y}_j \|}^2 \bm{w}_{ij} =
                \underbrace{
                    \sum_{i} {\| \bm{y}_i\|}^2 m_{ii} +
                    \sum_{j} {\| \bm{y}_j \|}^2 m_{jj}
                }_{2\bm{YMY^\top}} +
                \underbrace{
                    2\sum_{ij} (\bm{y}_i \bm{y}_j^\top ) \bm{w}_{ij}
                }_{2\bm{YWY^\top}},
        \end{equation}
        optimising this problem finds the same solution as the following:
        \begin{align}
            \min_{\bm{Y}} \quad & \bm{Y^\top LY} \\
            \textrm{s.t.} \quad 
                & \bm{Y^\top MY} = \bm{I_n},
        \end{align}
        where the constraint is needed to prevent the same trivial solution as in LLE.
        
        It can then be solved as the eigenvalue problem:
        \begin{equation}
            \bm{LV} = \bm{\lambda MV},
        \end{equation}
        where $\bm{L}\in\mathbb{R}^{D\times D}$ is the Laplacian matrix $\bm{L} = \bm{M}+\bm{W}$, selecting then the $d$ smallest eigenvectors and proceeding like explained before.

    \subsubsection{Hessian Locally Linear Embedding (H-LLE)}
        Considerately related to LLE, but instead of calculating the weighted best coordinates, in the embedded space, for a point from its k-nearest neighbourhood, the Hessian LLE \cite{h-lle} approximately calculates the curviness of the manifold from the local Hessian and minimises it in order to flatten the dataset.
        This is done simply by performing an eigenvalue decomposition over the Hessian Matrix of local Hessian approximations and looking for the $d$ smallest eigenvalues, the respective eigenvalues form the embedded $\bm{Y}$ matrix containing the low dimensional representation of the data.

        To find the matrix of Hessian estimations at each point, H-LLE assumes local linearity across the neighbourhood of each point and computes its tangent space by applying PCA to its $k$-nearest neighbors (collected into a matrix), ending up with selecting the $d$ most representative directions of the manifold around each data point.
        
        After that, the estimate of the Hessian at each point is formulated from the matrix $\bm{Z}$ which is composed of the cross product between the principal components, with an added column of $1$'s, at each point, and then orthogonalised through the Gram-Schmidt process.

        The local tangent Hessian estimation $\bm{H}_i$ is now built by transposing the selection of the last $\frac{d(d+1)}{2}$ columns of $\bm{Z}$, and the global $\bm{\mathcal{H}}_lm$ Hessian estimator consists of:
        \begin{equation}
            \bm{\mathcal{H}}_{lm} = \sum_{i} \sum_{j} \left((\bm{H}_i)_{jl} \times (\bm{H}_i)_{jm}\right).
        \end{equation}
        After estimating the tangent spaces on each point, H-LLE proceeds by minimising the curvature of the data set, described by $\bm{\mathcal{H}} \in \mathbb{R}^{n\times n}$, by solving the eigenvalue problem as in Eq. \ref{eigenvalue decomposition} and selecting the eigenvectors correspondent to the $d$ smallest non-zero eigenvalues results in the $\bm{Y} \in\mathbb{R}^{n\times d}$ embedded data.

        Importantly, the number $k$ of nearest neighbours around each point to approximate the Hessian must be such that $k > d (1 + \frac{1 + d}{2})$, where $d$ is the number of components found by PCA. This means that for regions of the manifold that are nonlinear enough, the number of components necessary to represent the data in a linear subspace may result in a very large $k$, which may fail in capturing local geometry. This seems to happen in practice, as demonstrated by experiments on natural datasets, which tend to lie on nonlinear manifolds \cite{h-lle}.
    
    \subsubsection{Local Tangent Space Analysis (LTSA)}
        Comparably to H-LLE, but linearising the data in a different manner, LTSA \cite{ltsa} describes its input data set as the tangent space of the manifold at each data point.
        
        With a different perspective, the local tangent space at each point will be approximated to its tangent space in the lower dimensionality.
        
        Fundamentally, the LTSA, from the local tangent space of the manifold at each data point, tries to find a mapping from that local tangent space, to the embedded position of the point in analysis.
        To put this in practice, the LTSA involves of the minimisation of differences between the possible embedded positions of a point, and the result after performing the mapping from the local tangent of the point to the lower dimensional space. This is formulated as the optimisation problem: 
        \begin{equation}
            \min_{\bm{Y},\bm{L}} \quad \sum_i \| \bm{y}_i \bm{J}_i - \bm{L}_i \bm{\Theta}_i \|^2,
        \end{equation}
        where $\bm{\Theta}_i$ represents the local tangent space calculated by the PCA over the manifold at the point $\bm{x}_i$, $\bm{L}$ is the map from the tangent of the manifold at the same point $\bm{x}_i$ to the objective point $\bm{y}_i$ and $\bm{J}_i$ is a double-centring matrix transformation equivalent to performing:
        \begin{equation}
            d_{ij} = -\frac{1}{2} \left( d_{ij} - \frac{1}{k} \sum_{l \in \mathcal{N}_i} d_{il} - \frac{1}{k} \sum_{l \in \mathcal{N}_i} d_{jl} + \frac{1}{k^2}\sum_{l,m \in \mathcal{N}_i} d_{lm} \right),
            \label{double centre}
        \end{equation}
        where $k$ is the size of the neighbourhood of $i$ and $\mathcal{N}_i$ represents the set of its points. This operation ensures that each row and column, of the resulting matrix, have a mean of $0$, and also, the whole matrix is mean $0$.
    
        This means that LTSA is concurrently looking, for each data point, for the final position and the map from the local tangent space to its final position.
        
        This optimisation method can also be solved as an eigenvalue problem over the alignment matrix $\bm{B}\in\mathbb{R}^{n\times n}$ which is iteratively computed from a matrix of 0's, where for the set $\mathcal{N}_i$ of nearest neighbours of $\bm{x}_i$ the matrix is updated following:
        \begin{equation}
            \bm{B}_{\mathcal{N}_i\mathcal{N}_i} = \bm{B}_{\mathcal{N}_{i-1}\mathcal{N}_{i-1}} + \bm{J}_k(\bm{I} - \bm{V}_i\bm{V}_i^\top)\bm{J}_k,
        \end{equation}
        where $\bm{V_iV_i}^\top$ represent the PCA projection calculated from the local neighbourhood of $i$, and subsequently, the local alignment of the manifold at the data point $i$ is represented by $\bm{I - V_iV_i}^\top$.
        
        The global alignment matrix $\bm{B}$ then represents the sum of all the local alignment matrices. Finishing the process, $\bm{B}$ is used to find the eigenbasis and subsequently the embedded data, by performing an eigenvalue decomposition over $0.5(\bm{B} + \bm{B}^\top)$ and selecting the eigenvectors corresponding to the $d$ smallest non-zero eigenvalues. \\        
    
    \subsubsection{t-distributed Stochastic Neighbour Embedding (t-SNE)}
        Originating from SNE \cite{sne}, t-SNE \cite{t-sne} is also most focused on data visualisation, that is, it is best fitted for dimensionality reduction into 2 or 3 dimensions. Both need a hand-picked final number of dimensions and capture the differences between data points by computing the conditional probability of a neighbourhood, $p_{i|j}$ and a similar $q_{i|j}$ for the reduced space, which can be compared with the weights from LE and other weight-based methods, but with a different way of calculating:
        \begin{align}
            p_{i|j} = \frac{\text{exp}(-\|\bm{x}_i-\bm{x}_i\|_D^2/2\sigma^2)}{\sum_{k\neq l}\text{exp}\left(-\| \bm{x}_k - \bm{x}_l \|_D^2/2\sigma^2 \right)}, &&
            q_{i|j} = \frac{\text{exp}(-\|\bm{y}_i-\bm{y}_i\|_d^2)}{\sum_{k\neq l}\text{exp}\left(-\| \bm{y}_k - \bm{y}_l \|_d^2 \right)}
        \end{align}
        
        Due to being asymmetric, these cause a problem on outlier points, where the distance to most points is very large, which causes the point to be too far away in the reduced space. t-SNE reduced its impact by using symmetric distances between each point $i$ and $j$ applying the mapping $p_{ij} = p_{ji} = \frac{p_{i|j} + p_{j|i}}{2n}$.
        
        Also, to better organise and visualise the data, instead of following a Gaussian distribution of the neighbour points over a given point $\bm{x}_i$, the t-SNE changes the formulation of $q_{ij}$ to follow a student t-distribution with one degree of freedom in order to spread the clusters further while not making the already separated ones too far:
        
        \begin{equation}
            q_{ij} = \frac{(1+\| \bm{y}_i - \bm{y}_j \|^2)^{-1}}{\sum_{k\neq l} (1+\| \bm{y}_k - \bm{y}_l \|^2)^{-1}}.
        \end{equation}
        
        This creates a much more balanced gradient function, avoiding situations where SNE would be biased to attract points that were already in seemingly good positions. 

        The following action is to minimise the sum of the Kullback-Leiber Divergence over all the points:
        \begin{equation}
            C = KL(P\|Q) \sum_i \sum_j p_{ij} \log\frac{p_{ij}}{q_{ij}},
        \end{equation}
        which, due to the symmetry of t-SNE, can be solved with a simple gradient function:
        \begin{equation}
            \frac{\partial C}{\partial \bm{y}_i} = 4 \sum_j (p_{ij} - q_{ij})(\bm{y}_{i} - \bm{y}_{j}).
        \end{equation}
        
        More importantly, this new probability distribution on the lower-dimensional space also approaches the inverse square law, meaning that, at far distances, the scale of the map becomes virtually irrelevant, making far-apart clusters of map points behave as a single point. 

        This leads us to the most impactful change in t-SNE: considering clusters of points as single points, the original $O(n^2)$ formulation can be made into $O(n \log(n))$. 

        Besides, from the fact that the whole structure of this method is focused on 2D and 3D representations of the data, this latest optimisation also decays with the increase of dimensions, combined with the fact that it is needed to know into how many dimensions we want t-SNE to reduce the data to, in cases of higher and harder to calculate intrinsic dimensions, t-SNE does not get so competitive comparatively with other DR methods.

    \subsubsection{Kernel PCA (K-PCA)}
        The Kernel PCA \cite{k-pca} is a method generalization from the classic PCA. Instead of searching for the best eigenbasis directly on the matrix presenting the relation between each pairwise variable, the K-PCA performs an initial transformation on the source data, following a kernel function. This operation can be related to a Classical MDS where the objective function, on an Euclidean basis, is transformed following the kernel transformation that is virtually able to unbend any non-linearity present in a manifold. Because K-PCA no longer uses the covariance or correlation values between variables, which naturally centre the data, now, the K-PCA uses a double-centring transformation like Eq. \ref{double centre}.

        After applying the kernel function and centring the data set, it ends up with an equivalent $\bm{X'}\in\mathbb{R}^{n\times D}$ matrix which can continue the PCA's solution as in Eq. \ref{eq: Sigma}: computing the covariance $\bm{\Sigma} = \bm{X}'\bm{X}'^\top$ and selecting the eigenvectors corresponding to the top $d$ eigenvalues.

        This method is not used more often because it is normally hard to find an adequate kernel function to linearise the dataset, since it is not usually found by any calculations, but rationalising, and some trial and error.
    
    \subsubsection{Maximum Variance Unfolding (MVU)}
    
    Maximum Variance Unfolding \cite{mvu} has a consequence that ended up fixing K-PCA's main flaw as stated before. Its optimisation problem can be observed as the search for a good lineariser kernel process: in basic terms, the reasoning behind MVU is to stretch any existing manifold that there might exist in the data set. This is done by maximizing the distance between the points of the data set, while maintaining local isometry between neighbour points.
    
    All this while keeping the data centred, to remove degenerate solutions, formulates:
    \begin{align}
        \max_{\bm{Y}} \quad & \sum_{ij} {\| \bm{y}_i - \bm{y}_j \|}^2 \\
        \textrm{s.t.} \quad 
            & \sum_i \bm{y}_i = 0\\
            & {\| \bm{y}_i - \bm{y}_j \|}^2 = {\| \bm{x}_i - \bm{x}_j \|}^2, \forall{i,j} \in\bm{G},
    \end{align}
    where the first constraint is responsible for keeping the data set centred while the following is the one which maintains the local distances after the linearisation of the manifold. $\bm{G}\in\mathbb{R}^{n\times n}$ is the neighbourhood matrix which represents whether $j$ is a $k$-nearest neighbour of $i$.
    
    Since this operation virtually flattens the manifold, this means that the Euclidean distance between two points in the manifold, after being stretched, is as close to the theoretical geodesic distance as it can be. Since this takes no approximation in order to find the real distance along a manifold, Isomap can be seen as an approximation to MVU.
    
    The difficulty of this problem comes from its non-convexity and the inexistence of an equivalent closed-form solution. With that, this optimisation problem can be converted\footnote{The conversion calculations and constraint relaxation from the non-convex to the SDP problem can be seen in the Appendix.} into an SDP, and thus convex problem \cite{mvu}:
    \begin{align}
        \max_{\bm{K}} \quad & \text{trace}(\bm{K}) \\
        \textrm{s.t.} \quad 
            & \sum_{ij} k_{ij} = 0 \\
            & \bm{K} \succeq 0 \\
            & k_{ii} + k_{jj} -2k_{ij} = {\| \bm{x}_i - \bm{x}_j \|}^2 \quad , \forall{i,j}\in\bm{G},
    \end{align}
    where the $\bm{K}\in\mathbb{R}^{n\times n}$ matrix and subsequently $k_{ij}\in\mathbb{R}$ represent the inner product between each the $i$-th and $j$-th points.

    When the solution is reached, the value of $\bm{K}$ is the learned kernel, which this process can be viewed as. After having the final problem's matrix, the dataset is virtually linearised,
    %and a linear method can be applied to reduce the dimensionality of the data set with minimal loss of information. A regular eigenvalue decomposition over the Gram matrix $\bm{K}$ can be then processed, as in Eq: \ref{eigenvalue decomposition}, to reach the reduced space.
    and so a regular eigenvalue decomposition can be performed as in Eq. \ref{eigenvalue decomposition} have a representation in the embedded space with minimal loss of information.

    As analysed in \cite{cube}, MVU presents a computational complexity of $O((nk)^3)$, i.e. the computational expenses grow cubically with the $n$ number of points and the $k$ number of neighbours to consider as the neighbourhood of each point, presenting a much faster growth compared to other methods. This is most influenced by the size of the Gram objective matrix, in $\mathbb{R}^{n\times n}$, of inner products between each pair of data points, and on the size of the neighbourhood to consider, creating a constraint for each pair of data points. Along with that, and common to all numerical approaches, solving the final eigenvalue problem has the computational complexity of $O(n^3)$. % Solving an SDP over a nxn semidefinite matrix is O(c n^3 + c^2 n^2 + c^3), where c=#constraints


















\section{Related Work}
    % relacionar, ou identificar a falta dela, entre trabalhos já existentes e o problema que vamos resolver
    % experiencias que suportam o uso do mvu e o quão util seria o mvu em disjoint manifolds? ou tmb que suportam o uso de non linear no geral sobre linear methods? ex o tal papaer do MRI que compara linear com non linear approches, no mvu mensioned in there

    Landmark MVU (L-MVU) \cite{l-mvu} is an extension to MVU which facilitates its application on larger data sets. It approximates the manifold's whole structure from a subset of reference points, the \textit{landmarks}, based on the original data. It was found that many of the constraints from MVU's formulation are redundant, so another improvement from this extension is the incremental insertion of those constraints, inserting more only when the solution takes invalid values.

    This extension was also proposed for related methods such as Isomap and LLE. Their landmark versions L-Isomap \cite{l-isomap, general-landmark} and L-LLE \cite{l-lle, general-landmark} result in speed improvements that allow its application to larger datasets at the cost of some precision.

    However, when dealing with natural datasets, greater problems emerge. Typically, natural datasets lie on multiple or disjoint manifolds. Furthermore, data may not be well sampled from the underlying manifolds, resulting in sparse areas in the data space. Both of these result in neighbourhood graphs that are not connected. Many of the methods above either perform poorly or have undefined solutions when applied to disconnected neighbourhood graphs.

    One solution to remedy this issue could be to embed a single component (e.g., the largest) and apply out-of-sample extensions to embed the others \cite{out-of-sample-1, out-of-sample-2, out-of-sample-3}. However, since out-of-extension methods rely on embedding each point as some function of other (relatively distant) points that are already embedded, it is likely that their distances in the data space are not very informative, yielding imprecise embeddings. Indeed, in practice we observe a degradation of the learned representations \cite{comparison}.

    Another solution might be to simply increase the value of $k$ until the resulting neighborhood graph is connected (with $k=n$ in the limit). However, using a large number of neighbors might result in the loss of local information, so that is not a good solution in practice \cite{comparison}.

    As such, \cite{inspiration} proposes adding edges between connected components of a (disconnected) graph such that the intrinsic dimensionality of the edge space is equal to that of each of the components. This is done iteratively between pairs of components until the NG becomes connected. They apply their method to LLE, H-LLE, LE, and Isomap, generally improving their abilities to perform NLDR on data lying on disjoint manifolds. However, this method has a few disadvantages:
    \begin{itemize}
        \item It assumes that all manifolds/connected components have the same intrinsic dimensionality.
        \item It relies on an estimation/prior knowledge of the intrinsic dimensionality of the data.
        \item It creates as many connections as possible between components. This is disadvantageous for methods such as MVU, where the rigidity introduced by the new edges may prevent the unfolding of the manifold.
    \end{itemize}


    %In practice, with the growth in size of each given dataset, some areas may appear where the distribution of data points becomes quite sparse compared to others. This, in addition to the neighbourhood graph system, can impact the method's performance. In these situations, it is possible that during the execution of the method, the representation of the data in these areas may get distorted, and in some extreme cases where the density of data points gets so lean that two different submanifolds are identified with no clear link connecting them. In these not-so-uncommon situations, disconnected neighbourhood graphs form disjoint manifolds.
    
    %On datasets that need those high number of connections, between data points, to keep those disjoint manifolds connected, due to the low density of points on some particular location of the manifold, or for the physical impossibility of data in between them, the MVU method doesn't perform well or can't even find a feasible solution.
    
    %This infeasibility is easily explained as the pulling motion that the MVU applies to the manifold. Without a connection keeping these submanifolds together, MVU separates them further into infinity. \\

    %It has been seen in practice that it is not reliable to connect the dataset solely by increasing the neighbourhood on each data point given that MVU and many other methods scale with the size of the k-neighbourhood \cite{comparison}.
    
    %With this in mind, in \cite{inspiration}, the solution for this problem was thought for Isomap. In the same, the approach to create connections between all the disjoint components, or taking an approach to connect the largest ones were considered.

    %Along with the problem of connecting the manifolds, the same problem can be thought of from another point of view. Instead of trying to connect all the components along a single manifold, another possibility is connecting them considering they take part into two distinct manifolds. In these situations, connecting them as if they made part of the same continuous manifolds could lead to misleading results. An approach to connect while accepting the differences in components is delved in \cite{modern} as an out-of-sample extension of MVU. This same approach could also take advantage of computing those multiple manifolds in parallel.
    
    %Multiple other extensions to MVU were also thought of in this same article \cite{modern}.

    % Onde é que falo nos principais usos para o MVU? tipo que o MVU é principalmente benéfico/usado para datasets que beneficiam altamente de menores números de variaveis.

    % Só no fim do rework é que me lembrei que tinhas dado a sujestão.
    To conclude, most non-linear dimensionality reduction methods rely on building a so-called neighbourhood graph (NG) of the data, which is used to discover its underlying geometric structure. These neighbourhood graphs can be highly disconnected if data are sampled from disjoint manifolds. Linear techniques such as PCA work fine in these cases, but NG-based methods do not work in these situations either because relationships between points are undefined (e.g., Isomap, where geodesic distances can't be calculated globally) or because the problem becomes ill-defined (e.g., MVU, where the optimization becomes unbounded).

\section{Proposed Approach}

\subsection{Problem Specification}
% MAIN SECTION: especificar com 0 ambiguidades que problema vamos resolver
    As mentioned before, NG-based methods have big performance downgrades, or don't work at all, on datasets that present some separation between clusters or submanifolds; ending up not being capable of finding a solution in cases where a connection between submanifolds is non-existent.

    Specifically for MVU, the problem originates from the fact that the constraints that take part in the optimisation problem are responsible for keeping the distances between points in its neighbourhood constant. In contrast, the overall optimisation function pulls every other point away. This results in an unbounded problem where MVU simply pulls the connected components further.
    
    In particular, while existing solutions have increased the applicability of other methods, they do not yet allow for the application of MVU to real-world data. Due to the stronger isometry properties of MVU and the fact that it usually outperforms other non-linear dimensionality reduction methods on artificial data \cite{comparison}, we wish to extend its applicability to real-world data.
    
\subsection{Solution Specification}

    In this subsection, we describe the procedure of our proposed solution:

    Consider a dataset of disjoint manifolds, where $\bm{S}_i \in \mathbb{R}^{n_i \times D}$ is the subset of points belonging to the connected component of index $i$, where $n_i$ is the number of points of the $i$'th disjoint component. %This can be calculated from a DFS or BFS algorithm.

    For each connected component $\bm{S}_i$ we preform MVU over it, in order to achieve an embedded representation $\bm{Y}_i \in \mathbb{R}^{n_i \times d}$.
    We now look for the subset $\bm{C}_i \in \mathbb{R}^{d+1 \times d}$ of $d+1$ points that are the furthest apart, which computationally would approximate to:
    \begin{equation}
        \max_{\bm{C}_i} \quad {\| \bm{c}_m - \bm{c}_n \|}^2, \forall{\bm{c}_m,\bm{c}_n} \in \bm{Y}_i.
    \end{equation}
    Additionally, we look for the points that are closest to each of the other connected component, forming the subset $\bm{L}\in \mathbb{R}^{l-1\times d}$, where l is the number of disjoint components:
    \begin{equation}
        \min_{\bm{L}_i} \quad {\| \bm{L}_{im} - \bm{L}_{jn} \|}^2, \forall \bm{L}_{im} \in \bm{Y}_{i} \wedge \bm{L}_{jn} \in \bm{Y}_{j}
    \end{equation}

    After performing MVU over each individual connected component, it is possibler to achieve embedded spaces with different sizes, thus it is necessary components to have the same dimensionality, adding variables of value $0$ until they equalise. 

    From each of the subsets $\bm{C}_i$ and $\bm{L}_i$ we now build a dataset of points from all the disjoint components, and preform MVU over it, now. It is important that the connection matrix of this formulation describes connections between all inner-component points, and the connections inter-component with the closest point between them:
    \begin{align}
        \max_{\bm{Y}} \quad & \sum_{ij} {\| \bm{y}_i - \bm{y}_j \|}^2 \\
        \textrm{s.t.} \quad 
            & \sum_i \bm{y}_i = 0 \\
            & {\| \bm{y}_i - \bm{y}_j \|}^2 = {\| \bm{C}_{im} - \bm{C}_{jn} \|}^2, \forall{C}_{im} \in\bm{C}_i \wedge  {C}_{jn} \in\bm{C}_j \\
            & {\| \bm{y}_i - \bm{y}_j \|}^2 = {\| \bm{L}_{im} - \bm{L}_{jn} \|}^2.
    \end{align}
    Due to MVU's constraint of keeping connected point' distances untouched, it is the possible to position each manifold in its place based on the reference points used to preform the inner-component reduction.


    \iffalse
    \begin{itemize}
        \item Preform a BFS or DFS search to compute the group of points forming the different submanifolds:
        \item Consider $i$ an index of each group of points forming a submanifold:\begin{itemize}
                \item Compute the shortest link from any point in the submanifold, to any point in \textcolor{red}{the closest manifold} 
                
                \item Preform MVU over each connected submanifold, reaching an embedded submanifold in $d_i$ dimensions.
                \item Find the $d+1$ furthest points
                \item Summarise each submanifold in a subset of points containing the $d+1$ furthest points and \textcolor{red}{all the points necessary for the inter-submanifold connections}.
            \end{itemize}
        \item Preform a basis equaliser, making all subset of submanifolds into d-dimensions. \textcolor{red}{yes they are all in $d$ dimensions, but are they the same $d$ dimensions?}
        \item Merge all the points from the subsets of manifolds into one dataset.
        \item Preform MVU over the newly created dataset.
        \item \textcolor{red}{Reconstruct each submanifold from each reference points chosen before.}
    \end{itemize}
    \fi

    

    %Different solutions for this problem can be considered:
    
    % The simpler one would be to store the centre point of each submanifold and perform the dimensionality reduction separately. This approach could return different eigenbasis for each of the manifolds, possibly leading to a difficult situation when positioning the different manifolds on the same basis.
    
    % Another possible approach, as explored for related methods like isomap, is to analyse the impact that creating random connections between points in different submanifolds would imply.

    % There is also the possibility of using a Neighbourhood graph with varying values of k depending on the density of the dataset in each particular location.
    % Some approaches with a variable value of k have been explored, but I'm not sure if anyone has already used them to ensure the dataset is fully connected.    This approach can be compared to an approach where we form the neighbourhood graph following one certain function that is not knn.
    
\subsection{Methodology}
% um paragrafo a explicar que o que vamos fazer nesta subsecção é descrever a metodologia para as experiências
Along with the need to measure the success of each method modification, there will also be the need to compare those method modifications with the other presented methods. Given that they present different interpretations and approaches to lowering the dimension of a dataset, the comparison between them may not take the same approach as comparing different versions of the same method.
With this said, below we present the various ways that we can compare the success of each method change:

\subsubsection{Algorithms}
% - enumerar os métodos que vamos testar
% - explicar o porquê de os escolhermos
% - explicar as configurações que vamos testar, i.e., para cada algoritmo dizer as combinações de hiperparâmetros (e.g., número de NNs) que vamos escolher nos testes (de forma a fazer uma comparação justa, porque algoritmos diferentes podem ser melhores ou piores para diferentes escolhas de hiperparâmetros)

We will test all the algorithms described above as they are the most widely used and are representative dimensionality reduction methods. Furthermore, we will augment them with the enhanced neighbourhood graph as in \cite{inspiration} to compare them and analyse their performance over datasets with disjoint manifolds.

% Due to its resemblance to MVU \cite{resemblance}, Isomap, LLE, LE, and K-PCA all appear to be valuable comparators with MVU. Aside from that, given the young success of t-SNE, although very focused on reducing dimensions to specific low values, it presents as a worthy competitor to MVU, along with LTSA and H-LLE.

Given that most methods in analysis use some type of neighbourhood graph, the $k$ hyper-parameter can be considered a unifying variable which would offer comparable situations to the set of methods, making it easy to observe the impact that disjoint datasets present to each method. However, this parameter may not take a rather large range of values, and is highly dependent on the properties of each dataset. Thus, alike \cite{comparison}, we are going to test the different datasets with $k$ ranging from $3$ to 15. Relatively to the learning rate, in the case of t-SNE, we are going to explore values between 10 and 1000.


\subsubsection{Datasets}
% - enumerar os datasets que vamos usar
% - explicar a relevância de cada dataset, i.e., porque é que interessa saber como é que os métodos se saem nesta tarefa
% - descrever os datasets em detalhe, e.g., resolução de imagens, origem do dataset no caso de um dataset de imagens naturais, ou no caso de datasets artificiais detalhar a forma como o geraste (como feito naquela review)

We will consider both artificial, which present unique challenges to the different algorithms, and natural datasets. To construct each artificial dataset, we consider the variables $p_i$ and $q_i$, uniformly distributed in the range $[0, 1]$. In artificial datasets, Gaussian noise with a small variance is added to all data points, as in \cite{comparison}.

\paragraph{Swiss Roll}
Addressing data that lies on a low-dimensional manifold that is isometric to Euclidean space, the Swiss Roll
represents a manifold turning into itself.

It can be built following simple computational operations leaving the possibility of some fine-tuning relating the density of points along the manifold. Being artificially built, the number of points can vary as we prefer, as can the density of points along the manifold.
However, the number of dimensions is expected to remain as the typical Swiss roll dataset, 3.

The Swiss Roll is thus formed of points $\bm{x}_i = [t_i \cos(t_i), t_i \sin(t_i), 30q_i]$ where $t_i = \frac{3\pi}{2}(1+2p_i)$.

\paragraph{Broken Swiss Roll} presents an extension of the previous dataset, presenting the cases where the density of points in some locations reach null points, separating the continuous Swiss roll manifold into multiple disjoint components. This, is aimed to analyse data that lies on or near a disconnected manifold. The formulation of this dataset is also based on the previous, with the added clause of rejecting the points where $\frac{2}{5} < t_i < \frac{4}{5}$, resulting in two disjoint components.

\paragraph{Helix} representing a closed coil in the span of a 3-dimensional space. Testing if each method handles data lying on a low-dimensional manifold that is not isometric to Euclidean space.

The dataset is built following $\bm{x}_i = [(2 + \cos(8p_i)) \cos(p_i),(2 + \cos(8p_i)) \sin(p_i), \sin(8p_i)]$.

\paragraph{Twinpeaks}
the dataset is formulated from $\bm{x}_i = [1 - 2p_i, \sin(\pi - 2\pi p_i)), \tanh(3 - 6q_i)]$.
This dataset also aims to evaluate data lying on a low-dimensional manifold that is
not isometric to Euclidean space.

\paragraph{High-Dimensional} dataset consists of points randomly sampled from a 5-dimensional manifold embedded in a 10-dimensional space.
This, aims to test data forming a manifold with a high intrinsic dimensionality by computing ten different combinations of the five uniformly distributed random variables, some of which are linear and some of which are nonlinear.\\

We now present the natural datasets to be used in our experiments:

\paragraph{Teapots} a 400 76x101 pixel images dataset, in 3-byte colours, of the same teapot from a 360-degree range of rotation.

\paragraph{MNIST} composed of up to 70000 20x20 pixel images of digits (0 through 9) written by hand, with a 256-bit grey scale.

\paragraph{UMIST} is a dataset of 564 images of 20 individuals' faces from different angles and poses. Each image is 220 x 220 pixels with a 256-bit grey scale.

It is expected that disconnections in neighbourhood graphs will happen either in regions of the data space between different digits and individuals, or in areas where the data is not well sampled.

These tasks are relevant examples to evaluate if the methods are able to identify some relation between data with no extra information besides the data itself.

\subsubsection{Metrics}
We follow \cite{comparison} and evaluate dimensionality reduction methods according to three metrics that assess the quality of the embedded data:

\paragraph{$1$-NN Classifier Error} \cite{1-nn} Calculates the ratio of points which, its closest neighbour is maintained from the original space throughout the reduction method.

\paragraph{Trustworthiness} \cite{t&c} Considering that $r(i,j)$ represents the position of $j$ in the ranking of the closest point to $i$ in the original space, and $\bm{U}_i$ limits that search to $j$'s that are neighbour of $i$ in the lower-dimension space but not in the original space, the trustworthiness asserts whether the points in the $k$-neighbourhood of an embedded point are also neighbours to it in the original space:
\begin{equation}
    T(k) = 1 - \frac{2}{nk(2n-3k-1)} \sum_i \sum_{j\in U_i^{(k)}} (r(i,j) - k).
\end{equation}

\paragraph{Continuity} \cite{t&c} Analogously, continuity asserts whether the points in the $k$-neighbourhood of a point in the original space are still neighbours to it in the embedded space:
\begin{equation}
    C(k) = 1 - \frac{2}{nk(2n-3k-1)} \sum_i \sum_{j\in V_i^{(k)}} (\hat{r}(i,j) - k),
\end{equation}
where $\hat{r}(i,j)$ represents the ranking of points in the neighbourhood of $i$ in the embedded space, and $\bm{V}_i$ the set of points in the neighbourhood of $i$ in the original space but not in the embedded space.


%In practice, with such big number of data points, it is hard for a method to comply perfectly with the nearest neighbour measure. Thus, the trustworthiness and continuity measures are built, basically weighing the difference of neighbours which stopped being neighbours and vice versa; data points, that should not have, ended up connecting in the reduced space. The formulas for the given metrics $C$ and $T$, for continuity and trustworthiness respectfully are:
%\begin{equation}
%    C(k) = 1 - \frac{2}{nk(2n-3k-1)} \sum_i \sum_{j\in V_i^{(k)}} (r(i,j) - k)
%\end{equation}
%\begin{equation}
%    T(k) = 1 - \frac{2}{nk(2n-3k-1)} \sum_i \sum_{j\in U_i^{(k)}} (\hat{r}(i,j) - k)
%\end{equation}
%where $r(i,j)$ represents the rank of the high-dimensional point j according to the pairwise distance between the high-dimensional data points, while $\hat{r}(i,j)$ in respect of the low-dimension; $V_i^{(k)}$ represents the set of points that are among the k nearest neighbours of $i$ on high the high dimension but not in the low dimension, and $U_i^{(k)}$ represents the same set but in the oppose order.

\subsubsection{Experimental Setup}
All algorithms and experiments described in this section shall be implemented in Python 3.12.5 \cite{python}, resorting to numerical (e.g., NumPy \cite{numpy}), scientific (e.g., SciPy \cite{scipy}), and optimization (e.g., CvxPy \cite{cvxpy,cvxpy-upgrade}) libraries whenever necessary.

Whenever possible, we will test our implementations of the relevant algorithms against existing publicly available ones, such as those in Scikit-Learn's \cite{scikit-learn} manifold learning package.

Finally, all experiments will run on GAIPS's Nexus and ADA1 servers.


\subsection{Work Schedule}

In this section we describe the work plan leading to the elaboration of a master's thesis.

\begin{figure}[htbp]
    \begin{center}
        \begin{ganttchart}[y unit title=0.4cm,
            y unit chart=0.5cm,
            x unit=0.49cm,
            vgrid,hgrid, 
            title label anchor/.style={below=-1.6ex},
            title left shift=.05,
            title right shift=-.05,
            title height=1,
            progress label text={},
            bar height=0.7,
            group right shift=0,
            group top shift=.6,
            group height=.3]{1}{22}
            %labels
            \gantttitle{Dissertation (2025)}{22} \\
            \gantttitle{Feb}{4}
            \gantttitle{Mar}{4}
            \gantttitle{Apr}{4}
            \gantttitle{May}{4}
            \gantttitle{Jun}{4}
            \gantttitle{Jul}{2} \\
            %tasks
            \ganttbar{Dataset collection}{1}{1} \\
            \ganttbar{Methods implementation}{2}{2} \\
            \ganttbar{Literature study}{3}{6} \\
            \ganttbar{Solution implementation}{7}{8}
            \ganttbar{}{10}{10}
            \ganttbar{}{12}{12} \\
            \ganttbar{Result assessment}{9}{9}
            \ganttbar{}{11}{11}
            \ganttbar{}{13}{13} \\
            \ganttbar{Document elaboration}{14}{22}
        \end{ganttchart}
    \end{center}
\caption{Gantt chart}
\label{fig:gantt}
\end{figure}

After having acquired the datasets, we plan to setup an initial implementation for the base methods and test them against existing implementations (e.g., those in the manifold learning package of Sklearn \cite{scikit-learn}).

We will dedicate some time to studying approaches that can be taken and consider if there are any other alternative implementations to test. From that study, we plan to put our original approach into practice and any alternative solution we may find, alternating between their implementation and the analysis of the results achieved, hopefully leading to iterative improvement.

After testing all the approaches considered, we will write a master's thesis about the study done.

%\section{Conclusions}

%%%%%%%%%%%%%%%%
% INTRODUCTION %
%%%%%%%%%%%%%%%%
%%%%%%%%%%%%%%
% Background %
%%%%%%%%%%%%%%
%%%%%%%%%%%%%%%%
% Related work %
%%%%%%%%%%%%%%%%
%%%%%%%%%%%%
% Solution %
%%%%%%%%%%%%
%%%%%%%%%%%%%%
% Evaluation %
%%%%%%%%%%%%%%
%%%%%%%%%%%%%%%%%
% Work Schedule %
%%%%%%%%%%%%%%%%%
%%%%%%%%%%%%%%%%
% BIBLIOGRAPHY %
%%%%%%%%%%%%%%%%

\clearpage
% add Bibliography to TOC
\addcontentsline{toc}{section}{\refname}
% The following command resets the 'emphasis' style for bibliography entries
\normalem
% Name of your BiBTeX file
\bibliography{Bibliography} % <-- ACTION: Put here your own filename or edit Bibliography.bib to add your references
% The following command modifies the 'emphasis' style for bibliography entries
\ULforem


%%%%%%%%%%%%
% APPENDIX %
%%%%%%%%%%%%
\clearpage % <-- ACTION: Comment the following blocks if you dont need appendixes
\appendix
% #############################################################################
% This is Appendix A
% !TEX root = ../main.tex
% #############################################################################
\chapter{Code of Project}
\label{chapter:appendixA}

Nulla dui purus, eleifend vel, consequat non, dictum porta, nulla. Duis ante mi, laoreet ut, commodo eleifend, cursus nec, lorem. Aenean eu est. Etiam imperdiet turpis. Praesent nec augue. Curabitur ligula quam, rutrum id, tempor sed, consequat ac, dui. Vestibulum accumsan eros nec magna. Vestibulum vitae dui. Vestibulum nec ligula et lorem consequat ullamcorper. 

\begin{lstlisting}[frame=lines,style=XML,caption={Example of a XML file.},label=xmlEx]
<?xml version="1.0" encoding="UTF-8"?>
<StreamInfo version="2.0">
    <Clip duration="PT01M0.00S">
        <BaseURL>videos/</BaseURL>
        <Description>svc_1</Description>
        <Representation mimeType="video/SVC" codecs="svc" frameRate="30.00" bandwidth="401.90"
            width="176" height="144" id="L0">
            <BaseURL>svc_1/</BaseURL>
            <SegmentInfo from="0" to="11" duration="PT5.00S">
                <BaseURL>svc_1-L0-</BaseURL>
            </SegmentInfo>
        </Representation>
        <Representation mimeType="video/SVC" codecs="svc" frameRate="30.00" bandwidth="1322.60"
            width="352" height="288" id="L1">
            <BaseURL>svc_1/</BaseURL>
            <SegmentInfo from="0" to="11" duration="PT5.00S">
                <BaseURL>svc_1-L1-</BaseURL>
            </SegmentInfo>
        </Representation>
    </Clip>
</StreamInfo>
\end{lstlisting}

Etiam imperdiet turpis. Praesent nec augue. Curabitur ligula quam, rutrum id, tempor sed, consequat ac, dui. Maecenas tincidunt velit quis orci. Sed in dui. Nullam ut mauris eu mi mollis luctus. Class aptent taciti sociosqu ad litora torquent per conubia nostra, per inceptos hymenaeos. Sed cursus cursus velit. Sed a massa. Duis dignissim euismod quam.


Class aptent taciti sociosqu ad litora torquent per conubia nostra, per inceptos hymenaeos. Phasellus eget nisl ut elit porta ullamcorper. Maecenas tincidunt velit quis orci. Sed in dui. Nullam ut mauris eu mi mollis luctus. Class aptent taciti sociosqu ad litora torquent per conubia nostra, per inceptos hymenaeos.

This inline MATLAB code \mcode{for i=1:3, disp('cool'); end;} uses the \verb|\mcode{}| command.\footnote{MATLAB Works also in footnotes: \mcodefn{for i=1:3, disp('cool'); end;}}

Nullam ut mauris eu mi mollis luctus. Class aptent taciti sociosqu ad litora torquent per conubia nostra, per inceptos hymenaeos. Sed cursus cursus velit. Sed a massa. Duis dignissim euismod quam. Nullam euismod metus ut orci.

\begin{lstlisting}[language=matlabfloz,caption={\mcode{Matlab Function}}]
for i = 1:3
	if i >= 5 && a ~= b       % literate programming replacement
		disp('cool');         % comment with some §\mcommentfont\LaTeX in it: $\mcommentfont\pi x^2$§
	end
	[:,ind] = max(vec);
	x_last = x(1,end) - 1;
	v(end);
	ylabel('Voltage (µV)');
end
\end{lstlisting}

Nullam ut mauris eu mi mollis luctus. Class aptent taciti sociosqu ad litora torquent per conubia nostra, per inceptos hymenaeos. Sed cursus cursus velit. Sed a massa. Duis dignissim euismod quam. Nullam euismod metus ut orci.

\lstinputlisting[
	label=lst:matlab_code,
	caption={\mcode{function.m}},
	breaklines=true
	]{./tables_and_code/function.m}

Class aptent taciti sociosqu ad litora torquent per conubia nostra, per inceptos hymenaeos. Phasellus eget nisl ut elit porta ullamcorper. Maecenas tincidunt velit quis orci. Sed in dui. Nullam ut mauris eu mi mollis luctus. Class aptent taciti sociosqu ad litora torquent per conubia nostra, per inceptos hymenaeos. Sed cursus cursus velit. Sed a massa. Duis dignissim euismod quam. Nullam euismod metus ut orci. Vestibulum erat libero, scelerisque et, porttitor et, varius a, leo.

\begin{lstlisting}[style=htmlcssjs,caption={HTML with CSS Code}]
<!DOCTYPE html>
<html>
  <head>
    <title>Listings Style Test</title>
    <meta charset="UTF-8">
    <style>
      /* CSS Test */
      * {
        padding: 0;
        border: 0;
        margin: 0;
      }
    </style>
    <link rel="stylesheet" href="css/style.css" />
  </head>
  <header> hey </header>
  <article> this is a article </article>
  <body>
    <!-- Paragraphs are fine -->
    <div id="box">			
			<p>
			  Hello World
			</p>
      <p>Hello World</p>
      <p id="test">Hello World</p>
			<p></p>
    </div>
    <div>Test</div>
    <!-- HTML script is not consistent -->
    <script src="js/benchmark.js"></script>
    <script>
      function createSquare(x, y) {
        // This is a comment.
        var square = document.createElement('div');
        square.style.width = square.style.height = '50px';
        square.style.backgroundColor = 'blue';
        
        /*
         * This is another comment.
         */
        square.style.position = 'absolute';
        square.style.left = x + 'px'; 
        square.style.top = y + 'px';
        
        var body = document.getElementsByTagName('body')[0];
        body.appendChild(square);
      };
      
      // Please take a look at +=
      window.addEventListener('mousedown', function(event) {
        // German umlaut test: Berührungspunkt ermitteln
        var x = event.touches[0].pageX;
        var y = event.touches[0].pageY;
        var lookAtThis += 1;
      });
    </script>
  </body>
</html>
\end{lstlisting}

Nulla dui purus, eleifend vel, consequat non, dictum porta, nulla. Duis ante mi, laoreet ut, commodo eleifend, cursus nec, lorem. Aenean eu est. Etiam imperdiet turpis. Praesent nec augue. Curabitur ligula quam, rutrum id, tempor sed, consequat ac, dui. Vestibulum accumsan eros nec magna. Vestibulum vitae dui. Vestibulum nec ligula et lorem consequat ullamcorper.

\begin{lstlisting}[style=htmlcssjs,caption={HTML CSS Javascript Code}]

@media only screen and (min-width: 768px) and (max-width: 991px) {
	
	#main {
		width: 712px;
		padding: 100px 28px 120px;
	}
	
	/* .mono {
		font-size: 90%;
	} */
	
	.cssbtn a {
		margin-top: 10px;
		margin-bottom: 10px;
		width: 60px;  
		height: 60px;   
		font-size: 28px;
		line-height: 62px;
	}
\end{lstlisting}

Nulla dui purus, eleifend vel, consequat non, dictum porta, nulla. Duis ante mi, laoreet ut, commodo eleifend, cursus nec, lorem. Aenean eu est. Etiam imperdiet turpis. Praesent nec augue. Curabitur ligula quam, rutrum id, tempor sed, consequat ac, dui. Vestibulum accumsan eros nec magna. Vestibulum vitae dui. Vestibulum nec ligula et lorem consequat ullamcorper.

\begin{lstlisting} [style=py,caption={PYTHON Code}]
class TelgramRequestHandler(object):
    def handle(self):
        addr = self.client_address[0]         # Client IP-adress
        telgram = self.request.recv(1024)     # Recieve telgram
        print "From: %s, Received: %s" % (addr, telgram)
        return
\end{lstlisting}

%%%%%%%%%%%%%
%    END    %
%%%%%%%%%%%%%

\end{document}
