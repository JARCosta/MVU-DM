% #############################################################################
% This is Chapter 1
% !TEX root = ../main.tex
% #############################################################################
% Change the Name of the Chapter i the following line
\fancychapter{Introduction}
% The following line allows to ref this chapter
\label{chap:intro}
\noindent \todo[color=green!40,author=Rui Cruz, inline]{The examples of techniques, tools, and packages along the document are for you to get familiarized with them. It is advisable to preserve those examples of usage, for reference, by moving the respective blocks of text to the last Chapter of this template (or to a Chapter file that you know you will not use), until you finish your document.}

\textcolor{violet}{Example of using package} \verb:todo: \textcolor{violet}{for notes of authors.} \textcolor{violet}{In this case} \todo[color=yellow!40,author=Johnny, fancyline]{pointing out to the place} \textcolor{violet}{the author Johnny is calling the attention for something at the specific place in the text.}

\textcolor{violet}{In this other case, another co-author is commenting on something inline.} \todo[color=orange!40,author=Manuel, inline]{Inline comment or Note. It can be an extract of some recommended text. ``...''}

\textcolor{violet}{In this other case, another co-author is making a note about the citation for missing some bibliographic record}~\cite{Apple:2011fk,AdobeHDS:ys,A.:qy}.
\todo[color=red!40,author=Pete]{You should cite also Pellentesque:2014}


Quisque facilisis erat a dui. Nam malesuada ornare dolor. \enquote{Cras gravida, diam sit amet rhoncus ornare, erat elit consectetuer erat, id egestas pede nibh eget odio.}\todo[color=green!40,author=Rui Cruz, fancyline]{notice here how to enquote correctly}{} 

\textcolor{violet}{This is an example of Tracking} \replaced[id=JR]{Changes}{Xanges} (in this case a replacement) by different authors in the document. The Text can additionally be modified by \added[id=FM]{adding} new text or by deleting \deleted[id=JB]{wrong} inadequate text. Author can manipulate changes \replaced[id=FM]{introduced by each author\deleted[id=JB]{, as adequate}}{intrroduced by other authors}.

% #############################################################################
\section{Morbi ipsum ipsum}
Praesent mauris \ac{SD} and \ac{HD} volutpat ligula eget enim \acp{WLAN} and 3G\slash 4G \acp{WWAN}.\todo[color=cyan!40, author=RC]{use of ACRONYMS that are defined in file ``Chapters/Thesis-MSc-Aconyms.tex''}

Aliquam erat \ac{WLAN} volutpat \ac{CPU} mauris nulla fermentum est \ac{OS} Fusce magna mi, porttitor quis, convallis eget, sodales ac, urna.
Pellentesque nibh felis, eleifend id, commodo in, interdum vitae, leo. Praesent eu elit. Ut eu ligula. Class aptent taciti sociosqu ad litora torquent per conubia nostra, per inceptos hymenaeos. Maecenas elementum augue nec nisl. Please notice the use of automatic referencig to objects such as Figures, Tables, equations, Algorithms, sections of a document, etc. by using the command \verb:\Cref{ref}: as in this case pointing to \Cref{fig:cashed}.\todo[color=cyan!40, author=RC, fancyline]{the correct Name of the float object, in this case a Figure, is determined by the system}

\begin{figure}[h]
\centering
\includegraphics[width=0.9\textwidth]{./Images/template/cashed5}
\caption{Ecosystem}
\label{fig:cashed}
\end{figure}

Proin bibendum gravida nisl. Fusce lorem. Phasellus sagittis, nulla in hendrerit laoreet, libero lacus feugiat urna, eget hendrerit pede magna vitae lorem. Praesent mauris Class aptent taciti sociosqu ad litora torquent per conubia nostra, per inceptos hymenaeos H.264\slash \ac{AVC} standard, sem vitae rhoncus tristique \ac{SVC} \cite{Fraunhofer-Heinrich-Hertz-Institute:2013fk,ISO:H-264} nulla in hendrerit laoreet, libero lacus feugiat urna, eget hendrerit pede magna vitae lorem.

\textcolor{violet}{You can use in-paragraph lists with this construct for: 
\begin{inparaenum}[(a)]
\item first case;
\item second case; and
\item third case,
\end{inparaenum}
making the text organized and fluid.}

Notice that \gls{maths} makes extensive use of \Glspl{formula}\todo[color=cyan!40, author=RC, fancyline]{example of use of Glossaries}{} which are particularly well rendered in documents produced with \gls{LaTeX}.

% #############################################################################
\section{Organization of the Document}
This thesis is is organized as follows: \Cref{chap:intro} \todo[color=cyan!40, author=RC, fancyline]{references to doc sections/chapters are automatic}{}interdum vel, tristique ac, condimentum non, tellus. 
In \cref{chap:back} curabitur nulla purus, feugiat id, elementum in, lobortis quis, pede.
In \cref{chap:architecture} consequat ligula nec tortor. Integer eget sem. Ut vitae enim eu est vehicula gravida.
\Cref{chap:implement} morbi egestas, urna non consequat tempus, nunc arcu mollis enim, eu aliquam erat nulla non nibh in \cref{chap:evaluation}.
\Cref{chap:conclusion} suspendisse dolor nisl, ultrices at, eleifend vel, consequat at, dolor.